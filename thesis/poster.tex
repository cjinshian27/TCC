% Poster get from https://github.com/victorsenam/tcc/blob/master/poster/main.tex

\documentclass[final]{beamer}
\usepackage[size=a1,orientation=portrait,scale=1.3]{beamerposter}

\usepackage[brazil]{babel}
\usepackage[utf8]{inputenc}
\usepackage[T1]{fontenc}
\usepackage{framed,graphicx,xcolor} % for shaded box
\usepackage{mathtools}%
\usepackage{tikz}
\usetikzlibrary{matrix,shapes,positioning,shadows,trees,patterns}

\usepackage[shortlabels]{enumitem}
\usepackage[numbers]{natbib}
\bibliographystyle{plainnat}
  \def\bibfont{\small}

\sloppy

%----------------------------------------------------------------------------------------
%	SHORTCUTS
%----------------------------------------------------------------------------------------
\newcommand{\B}[1]{\mathbb{#1}}
\newcommand{\Cl}[1]{\ensuremath{\mathcal{#1}}}

\newcommand{\sse}{\Leftrightarrow}
\newcommand{\so}{\Rightarrow}
\newcommand{\se}{\Leftarrow}
\newcommand{\rec}{\leftarrow}

\newcommand{\tdots}{\,.\,.\,}

%----------------------------------------------------------------------------------------
%	BEAMER STYLE
%----------------------------------------------------------------------------------------

\usetheme{poster}
\setbeamercolor{block title}{fg=dblue,bg=white}
\setbeamercolor{block body}{fg=black,bg=white}
\setbeamercolor{block alerted title}{fg=dblue,bg=gray!50}
\setbeamercolor{block alerted body}{fg=black,bg=gray!20}
\setbeamercolor{block prob}{fg=black,bg=white}
\setbeamertemplate{caption}[numbered]

%----------------------------------------------------------------------------------------
%	CUSTOM STYLING
%----------------------------------------------------------------------------------------

\newenvironment<>{prob}{
    \begin{beamercolorbox}[sep=1ex,center,dp={1ex}]{block prob}
    \textcolor{dblue}{\textbf{Problema:}}\itshape
}{\end{beamercolorbox}}

\newcommand\halfcol{\column{.46\textwidth}}
\newcommand\onethirdcol{\column{.31\textwidth}}

\newcommand{\Oh}{\mathrm{O}}

% ?????????
\usepackage{subcaption}

\newcommand*\bolinha[1]{\; \tikz[inner sep=.25ex]\node[circle,draw]{#1}; \;}

%----------------------------------------------------------------------------------------
%	POSTER
%----------------------------------------------------------------------------------------

\title{Florestas geradoras maximais de custo
mínimo em grafos dinâmicos}
\author{Chung Jin Shian \hspace{200pt} Orientadora: Cristina Gomes Fernandes}
\institute{\vspace{10pt}Departamento de Ciência da Computação,
Instituto de Matemática e Estatística, Universidade de São Paulo}


\begin{document}
\begin{frame}[fragile]\centering
  \vspace{-40pt}
  \begin{columns}[T]

    % ----------------------------------------------------------------------------------------
    % PRIMEIRA COLUNA
    % ----------------------------------------------------------------------------------------
    \onethirdcol
    \begin{alertblock}{Resumo}
      Grafos dinâmicos permitem modelar problemas em que o conjunto de arestas do grafo sofre alterações ao longo do tempo. Um dos problemas fundamentais nesse contexto a manutenção de uma árvore geradora de custo mínimo do grafo dinâmico. Neste trabalho, estudamos vários algoritmos propostos por Holm, de Lichtenberg e Thorup \cite{jacob_holm} para esse variantes desse problema. O foco do trabalho foi no algoritmo manter uma floresta maximal de custo mínimo (MSF) decremental, ou seja, que dá suporte eficiente e remoção de arestas. Esse algoritmo foi implementado e testado em grafos com centenas de milhares de vértices. 
    \end{alertblock}

    \begin{block}{Conexidade em grafos dinâmicos}
     O problema da conexidade em grafos dinâmicos visa uma implementação eficiente dos métodos da biblioteca abaixo:
      
            \definecolor{shadecolor}{rgb}{0.93, 0.80, 0.82} % pale pink
      \begin{shaded}
        \begin{itemize}
            \item[$\bullet$] \texttt{\textbf{grafoDinâmico($n$)}}: contrói e devolve um grafo dinâmico com $n$ vértices e sem arestas;
            \item[$\bullet$] \texttt{\textbf{conectadosGD($G$, $u$, $v$)}}: devolve verdadeiro se os vértices $u$ e $v$ estão na mesma componente de $G$ e falso caso contrário;
            \item[$\bullet$] \texttt{\textbf{adicioneGD($G$, $u$, $v$)}}: adiciona a aresta $uv$ no grafo $G$;
            \item[$\bullet$] \texttt{\textbf{removaGD($G$, $u$, $v$)}}: remove a aresta $uv$ do grafo $G$.
        \end{itemize} 
      \end{shaded}

      \definecolor{shadecolor}{rgb}{0.74, 0.83, 0.9} % pale blue
      \begin{shaded}
        \textbf{Ideia:} fatiar o grafo $G$ em níveis. Cada aresta do grafo possui um nível entre 1 e $\left\lceil \lg n \right\rceil$, onde $n$
        é o número de vértices do grafo $G$. Uma aresta, ao ser inserida em $G$, começa com o nível $\left\lceil \lg n \right\rceil$. Seja $G_i$ = $G[X]$ onde $X$ é o conjunto das arestas do grafo $G$ de nível menor ou igual a $i$. Para cada nível $i$, o algoritmo mantém uma floresta maximal $F_i$ de $G_i$. Além disso, ele mantém também, para cada nível $i$, um grafo $R_i$ em forma de listas de adjacências, que guardam as arestas de nível $i$ que não estejam em $F_i$, chamadas de aresta reserva.
      \end{shaded}
      

      Cada floresta $F_i$ é mantida em uma estrutura de dados específica para florestas dinâmicas, baseada em Euler tour trees.
      \bigskip
      
\begin{figure}[H]
    \centering

    % Elemento 1: Grafo original e Dígrafo Euleriano (do primeiro figure)
    \begin{minipage}{0.7\textwidth} % Usa aproximadamente 70% da largura do texto
        \centering
        \begin{tikzpicture}[thick, node distance=2cm,
          vtx/.style={draw, circle, minimum size=10mm, inner sep=0pt}]

        % ---------- Left: original graph ----------
        \node[vtx] (a) at (0,0) {$a$};
        \node[vtx] (b) at (1.5,1.5) {$b$};
        \node[vtx] (c) at (3,0) {$c$};
        \node[vtx] (d) at (1.5,-1.5) {$d$};
        \node[vtx] (e) at (-1.5,-1.5) {$e$};

        % Graph edges
        \draw (a) -- (b);
        \draw (c) -- (d);
        \draw (d) -- (a);
        \draw (a) -- (e);
        
        % Adiciona uma label ou descrição para o primeiro gráfico
        \node at (-1, 1) {$T$};

        % ---------- Right: Euler-tour representation ----------
        \begin{scope}[xshift=8cm]
          \node[vtx] (a1) at (0,0) {$a$};
          \node[vtx] (b1) at (1.5, 1.5) {$b$};
          \node[vtx] (c1) at (3,0) {$c$};
          \node[vtx] (d1) at (1.5,-1.5) {$d$};
          \node[vtx] (e1) at (-1.5,-1.5) {$e$};

          % Bidirectional arcs
          \draw[->] (a1) to[bend left=20] (b1);
          \draw[->] (b1) to[bend left=20] (a1);

          \draw[->] (c1) to[bend left=20] (d1);
          \draw[->] (d1) to[bend left=20] (c1);

          \draw[->] (d1) to[bend left=20] (a1);
          \draw[->] (a1) to[bend left=20] (d1);

          \draw[->] (a1) to[bend left=20] (e1);
          \draw[->] (e1) to[bend left=20] (a1);

          \draw[->] (a1) edge[loop above] ();
          \draw[->] (b1) edge[loop above] ();
          \draw[->] (c1) edge[loop right] ();
          \draw[->] (d1) edge[loop below] ();
          \draw[->] (e1) edge[loop left] ();
          
          % Adiciona uma label ou descrição para o segundo gráfico
          \node at (1.5, 2.5) {};

        \end{scope}
        
        \draw[->, blue, very thick] ([xshift=1cm]b.east) 
            to[out=30, in=150] ([xshift=6cm]b.west); % Start from 'b.east', 
        \node[blue, font=\small] at (5.5, 2.6) {digrafo Euleriano}; % Text above the arrow path
          
         \draw[->, blue, very thick] ([yshift=-1cm]d1.south) 
            to[out=-20, in=20] ([yshift=-5cm]d1.south); % Start from 'b.east', 
          \node[blue, font=\small, fill=white, inner sep=2pt] at (10.5, -4.5) {trilha};
          \node[blue, font=\small, fill=white, inner sep=2pt] at (10.5, -5) {Euleriana};

          

        \end{tikzpicture}
    \end{minipage}

    \vspace{-2cm}
    \begin{minipage}{\textwidth} 
        \centering
        \begin{tikzpicture}
              [node/.style={circle,draw,minimum size=10mm, thick},
              edge/.style={thick, black},
              reserve/.style={red, thick},
              removed/.style={black, thick, dashed}, 
              inner sep=0pt]

              % Aumentando a distância horizontal (xshift) e vertical (yshift)
              % para melhor visualização ao combinar com o primeiro gráfico, 
              % e ajustando as coordenadas para centralizar o nó raiz.
              
              \node[node] (ad) at (0, 0) {$ad$};
              \node[node] (ab) at (-4.5, -1.5) {$ab$};
              \node[node] (cd) at (4.5, -1.5) {$cd$};
              \node[node] (dc) at (3, -3) {$dc$};
              \node[node] (dd) at (1.5, -4.5) {$dd$};
              \node[node] (cc) at (4.5, -4.5) {$cc$};
              \node[node] (da) at (6, -3) {$da$};
              \node[node] (ae) at (7.5, -4.5) {$ae$};
              
              
              \node[node] (ea) at (-6, -3) {$ea$};
              \node[node] (ee) at (-7.5, -4.5) {$ee$};
              \node[node] (aa) at (-4.5, -4.5) {$aa$};


              \node[node] (bb) at (-3, -3) {$bb$};
              \node[node] (ba) at (-1.5, -4.5) {$ba$};

              

              % tree edges (normal black edges)
              \draw[edge] (ad) -- (ab) node[midway, below] {};
              \draw[edge] (ad) -- (cd) node[midway, below] {};
              \draw[edge] (cd) -- (dc) node[midway, below] {};
              \draw[edge] (cd) -- (da) node[midway, below] {};
              \draw[edge] (bb) -- (ba) node[midway, below] {};
              \draw[edge] (ea) -- (ee) node[midway, below] {};
              \draw[edge] (ea) -- (aa) node[midway, below] {};
              \draw[edge] (dc) -- (dd) node[midway, below] {};
              \draw[edge] (dc) -- (cc) node[midway, below] {};
              \draw[edge] (da) -- (ae) node[midway, below] {};

              \draw[edge] (ab) -- (ea) node[midway, below] {};
              \draw[edge] (ab) -- (bb) node[midway, below] {};

              \node at (-4, 0.9) {Euler tour};
              \node at (-4, 0) {tree de $T$};
        \end{tikzpicture}
        % Adiciona uma label ou descrição para o terceiro gráfico
    \end{minipage}

    \caption{Árvore $T$ e a Euler tour tree de $T$}
    \label{fig:arvore-euler-tour-combinada}
\end{figure}


    \end{block}





















    \begin{block}{Conexidade em florestas dinâmicas} 
      H
    \end{block}

    
    % ----------------------------------------------------------------------------------------
    % SEGUNDA COLUNA
    % ----------------------------------------------------------------------------------------
    \onethirdcol
    
    \begin{block}{Conexidade em grafos dinâmicos}

      
      Se os vértices $u$ e $v$ não estiverem na mesma componente de $G$, então a aresta $uv$ é inserida na floresta maximal $F_L$, assim ligando a árvore que contém $u$ com a que contém $v$ em $F_L$ . Chamamos esse tipo de aresta de \textbf{aresta da floresta}. Caso $u$ e $v$ já estejam conectados em $G$, então essa aresta $uv$ é chamada de \textbf{aresta reserva} e ela será armazenada no grafo $R_L$.

      \begin{figure}[H]
        \centering

        \noindent
        \begin{minipage}[c]{2cm}
            \raggedright
            Nível $4$
        \end{minipage}%
        \begin{minipage}[c]{0.8\textwidth}
            \centering
            \begin{tikzpicture}
            [node/.style={scale=0.7, circle,draw,minimum size=2em, thick, font=\large},
            edge/.style={thick, black},
            reserve/.style={red, thick},
            removed/.style={black, thick, dashed}]

            \node[node] (u) at (-1,2) {$u$};
            \node[node] (a) at (-3,4) {$a$};
            \node[node] (b) at (-5,2) {$b$};
            \node[node] (c) at (-3,0) {$c$};
            \node[node] (v) at (2,2) {$v$};
            \node[node] (d) at (3,3.7) {$d$};
            \node[node] (e) at (3,0.3) {$e$};
            \node[node] (f) at (5,0.3) {$f$};
            \node[node] (g) at (6, 2) {$g$};
            \node[node] (h) at (5, 3.7) {$h$};
            
            % Dotted circles for T_u and T_v
            \draw[dotted, thick, gray] (-3,2) circle (2.7cm);
            \draw[dotted, thick, gray] (4,2) circle (2.7cm);  
            
            % Labels for the circles
            \node at (-3,-1.1) {$T_u$};
            \node at (4,-1.1) {$T_v$};

            % tree edges (normal black edges)
            \draw[edge] (a) -- (u) node[midway, below] {};
            \draw[edge] (a) -- (b) node[midway, below] {};
            \draw[edge] (b) -- (c) node[midway, below] {};
            \draw[edge] (v) -- (d) node[midway, below] {};
            \draw[edge] (v) -- (f) node[midway, below] {};
            \draw[edge] (v) -- (e) node[midway, below] {};
            \draw[edge] (f) -- (g) node[midway, below] {};
            \draw[edge] (g) -- (h) node[midway, below] {};
            % reserve edges (normal red edges)
            
            \draw[removed] (u) -- (v) node[midway, below] {};
            \draw[reserve] (a) -- (d) node[midway, below] {};
            \draw[reserve] (c) -- (v) node[midway, below] {};
            \draw[reserve] (b) -- (u) node[midway, below] {};
            \draw[reserve] (c) -- (u) node[midway, below] {};
            \draw[reserve] (d) -- (f) node[midway, below] {};
            \draw[reserve] (d) -- (h) node[midway, below] {};

            \end{tikzpicture}
        \end{minipage}
        \vspace{1cm}
            \noindent
        \begin{minipage}[c]{2cm}
            \raggedright
            Nível $3$
        \end{minipage}%
        \begin{minipage}[c]{0.8\textwidth}
            \centering
            \begin{tikzpicture}
                [node/.style={scale=0.7, circle,draw,minimum size=2em, thick, font=\large},
                edge/.style={thick, black},
                reserve/.style={red, thick},
                removed/.style={black, thick, dashed}]

                \node[node] (u) at (-1,2) {$u$};
                \node[node] (a) at (-3,4) {$a$};
                \node[node] (b) at (-5,2) {$b$};
                \node[node] (c) at (-3,0) {$c$};
                \node[node] (v) at (2,2) {$v$};
                \node[node] (d) at (3,3.7) {$d$};
                \node[node] (e) at (3,0.3) {$e$};
                \node[node] (f) at (5,0.3) {$f$};
                \node[node] (g) at (6, 2) {$g$};
                \node[node] (h) at (5, 3.7) {$h$};
                
                %\draw[edge] (a) -- (u) node[midway, below] {};
                %\draw[edge] (a) -- (b) node[midway, below] {};
                %\draw[edge] (b) -- (c) node[midway, below] {};

            \end{tikzpicture}
        \end{minipage}
        \caption{Um grafo $G$ de 10 vértices, onde as arestas pretas são da floresta $F_4$, enquanto as vermelhas são reservas. A aresta $uv$ está prestes a ser removida. A floresta $F_{4}$ de $G$ de cima contém todas as arestas pretas recém-inseridas e as arestas vermelhas estão em $R_4$. A floresta de baixo é a $F_{3}$, com os vértices isolados, e $R_3$ também não tem nenhuma aresta.}
        \label{fig:example-replacement1}
    \end{figure}

\begin{figure}[H]
    \centering

    \noindent
    \begin{minipage}[c]{2cm}
        \raggedright
        Nível $4$
    \end{minipage}%
    \begin{minipage}[c]{0.8\textwidth}
        \centering
        \begin{tikzpicture}
        [node/.style={scale=0.7, circle,draw,minimum size=2em, thick, font=\large},
        edge/.style={thick, black},
        reserve/.style={red, thick},
        removed/.style={black, thick, dashed}]

        \node[node] (u) at (-1,2) {$u$};
        \node[node] (a) at (-3,4) {$a$};
        \node[node] (b) at (-5,2) {$b$};
        \node[node] (c) at (-3,0) {$c$};
        \node[node] (v) at (2,2) {$v$};
        \node[node] (d) at (3,3.7) {$d$};
        \node[node] (e) at (3,0.3) {$e$};
        \node[node] (f) at (5,0.3) {$f$};
        \node[node] (g) at (6, 2) {$g$};
        \node[node] (h) at (5, 3.7) {$h$};
        
         % Dotted circles for T_u and T_v
        \draw[dotted, thick, gray] (-3,2) circle (2.7cm);
        \draw[dotted, thick, gray] (4,2) circle (2.7cm);  
        
        % Labels for the circles
        \node at (-3,-1.1) {$T_u$};
        \node at (4,-1.1) {$T_v$};

        % tree edges (normal black edges)
        \draw[edge] (a) -- (u) node[midway, below] {};
        \draw[edge] (a) -- (b) node[midway, below] {};
        \draw[edge] (b) -- (c) node[midway, below] {};
        \draw[edge] (v) -- (d) node[midway, below] {};
        \draw[edge] (v) -- (f) node[midway, below] {};
        \draw[edge] (v) -- (e) node[midway, below] {};
        \draw[edge] (f) -- (g) node[midway, below] {};
        \draw[edge] (g) -- (h) node[midway, below] {};
        % reserve edges (normal red edges)
        
        \draw[reserve] (a) -- (d) node[midway, below] {};
        \draw[reserve] (c) -- (v) node[midway, below] {};
        \draw[reserve] (b) -- (u) node[midway, below] {};
        \draw[reserve] (c) -- (u) node[midway, below] {};
        \draw[reserve] (d) -- (f) node[midway, below] {};
        \draw[reserve] (d) -- (h) node[midway, below] {};

        \end{tikzpicture}
    \end{minipage}
    \vspace{1cm}
        \noindent
    \begin{minipage}[c]{2cm}
        \raggedright
        Nível $3$
    \end{minipage}%
    \begin{minipage}[c]{0.8\textwidth}
        \centering
        \begin{tikzpicture}
            [node/.style={scale=0.7, circle,draw,minimum size=2em, thick, font=\large},
            edge/.style={thick, black},
            reserve/.style={red, thick},
            removed/.style={black, thick, dashed}]

            \node[node] (u) at (-1,2) {$u$};
            \node[node] (a) at (-3,4) {$a$};
            \node[node] (b) at (-5,2) {$b$};
            \node[node] (c) at (-3,0) {$c$};
            \node[node] (v) at (2,2) {$v$};
            \node[node] (d) at (3,3.7) {$d$};
            \node[node] (e) at (3,0.3) {$e$};
            \node[node] (f) at (5,0.3) {$f$};
            \node[node] (g) at (6, 2) {$g$};
            \node[node] (h) at (5, 3.7) {$h$};
            
            
            \draw[edge] (a) -- (u) node[midway, below] {};
            \draw[edge] (a) -- (b) node[midway, below] {};
            \draw[edge] (b) -- (c) node[midway, below] {};

        \end{tikzpicture}
    \end{minipage}
    \caption{Representação da remoção da aresta $uv$ em $G$. As arestas de nível $4$ de $T_u$ foram rebaixadas para o nível $3$, o que pode ser visto na floresta $F_{3}$.}
    \label{fig:example-replacement2}    
\end{figure}

    \end{block}































    % ----------------------------------------------------------------------------------------
    % TERCEIRA COLUNA
    % ----------------------------------------------------------------------------------------
    \onethirdcol

        \begin{figure}[H]
    \centering

    \noindent
    \begin{minipage}[c]{2cm}
        \raggedright
        Nível $4$
    \end{minipage}%
    \begin{minipage}[c]{0.8\textwidth}
        \centering
        \begin{tikzpicture}
        [node/.style={scale=0.7, circle,draw,minimum size=2em, thick, font=\large},
        edge/.style={thick, black},
        reserve/.style={red, thick},
        removed/.style={black, thick, dashed}, 
        reserveremoved/.style={red, thick, dashed}]

        \node[node] (u) at (-1,2) {$u$};
        \node[node] (a) at (-3,4) {$a$};
        \node[node] (b) at (-5,2) {$b$};
        \node[node] (c) at (-3,0) {$c$};
        \node[node] (v) at (2,2) {$v$};
        \node[node] (d) at (3,3.7) {$d$};
        \node[node] (e) at (3,0.3) {$e$};
        \node[node] (f) at (5,0.3) {$f$};
        \node[node] (g) at (6, 2) {$g$};
        \node[node] (h) at (5, 3.7) {$h$};
        
         % Dotted circles for T_u and T_v
        \draw[dotted, thick, gray] (-3,2) circle (2.7cm);
        \draw[dotted, thick, gray] (4,2) circle (2.7cm);  
        
        % Labels for the circles
        \node at (-3,-1.1) {$T_u$};
        \node at (4,-1.1) {$T_v$};

        % tree edges (normal black edges)
        \draw[edge] (a) -- (u) node[midway, below] {};
        \draw[edge] (a) -- (b) node[midway, below] {};
        \draw[edge] (b) -- (c) node[midway, below] {};
        \draw[edge] (v) -- (d) node[midway, below] {};
        \draw[edge] (v) -- (f) node[midway, below] {};
        \draw[edge] (v) -- (e) node[midway, below] {};
        \draw[edge] (f) -- (g) node[midway, below] {};
        \draw[edge] (g) -- (h) node[midway, below] {};
        % reserve edges (normal red edges)
  
        \draw[reserve] (a) -- (d) node[midway, below] {};
        \draw[reserve] (c) -- (v) node[midway, below] {};
        \draw[reserveremoved] (b) -- (u) node[midway, below] {};
        \draw[reserveremoved] (c) -- (u) node[midway, below] {};
        \draw[reserve] (d) -- (f) node[midway, below] {};
        \draw[reserve] (d) -- (h) node[midway, below] {};

        \end{tikzpicture}
    \end{minipage}
    \vspace{1cm}
        \noindent
    \begin{minipage}[c]{2cm}
        \raggedright
        Nível $3$
    \end{minipage}%
    \begin{minipage}[c]{0.8\textwidth}
        \centering
        \begin{tikzpicture}
            [node/.style={scale=0.7, circle,draw,minimum size=2em, thick, font=\large},
            edge/.style={thick, black},
            reserve/.style={red, thick},
            removed/.style={black, thick, dashed}]

            \node[node] (u) at (-1,2) {$u$};
            \node[node] (a) at (-3,4) {$a$};
            \node[node] (b) at (-5,2) {$b$};
            \node[node] (c) at (-3,0) {$c$};
            \node[node] (v) at (2,2) {$v$};
            \node[node] (d) at (3,3.7) {$d$};
            \node[node] (e) at (3,0.3) {$e$};
            \node[node] (f) at (5,0.3) {$f$};
            \node[node] (g) at (6, 2) {$g$};
            \node[node] (h) at (5, 3.7) {$h$};
            
            
            % Dotted circles for T_u and T_v
            \draw[dotted, thick, gray] (-3,2) circle (2.7cm);
            \draw[dotted, thick, gray] (4,2) circle (2.7cm);  
            
            % Labels for the circles
            \node at (-3,-1.1) {$T_u$};
            \node at (4,-1.1) {$T_v$};
            
            \draw[reserve] (b) -- (u) node[midway, below] {};
            \draw[reserve] (c) -- (u) node[midway, below] {};
            \draw[edge] (a) -- (u) node[midway, below] {};
            \draw[edge] (a) -- (b) node[midway, below] {};
            \draw[edge] (b) -- (c) node[midway, below] {};

        \end{tikzpicture}
    \end{minipage}
    \caption{Representação da busca por uma aresta substituta em $R_4$. As arestas reserva de nível $4$ que estão tracejadas foram percorridas e estão prestes a ser removidas de $R_4$, pois foram rebaixadas para o nível $3$, como se pode ver em $R_3$.}
    \label{fig:example-replacement3}
\end{figure}

    \begin{block}{MSF decremental}
      


      \definecolor{shadecolor}{rgb}{0.93, 0.80, 0.82} % pale pink
      \begin{shaded}
        \begin{itemize}
          \item[$\bullet$] \texttt{\textbf{MSFDecremental($n$, $E$)}}: contrói e devolve um grafo ponderado $G$ com $n$ vértices e as arestas ponderadas dadas no conjunto $E$;
          \item[$\bullet$] \texttt{\textbf{consultePesoMSF($G$)}}: devolve o peso de uma MSF do grafo ponderado $G$;
          \item[$\bullet$] \texttt{\textbf{removaMSF($G$, $u$, $v$)}}: remove a aresta $uv$ do grafo ponderado $G$.
        \end{itemize}
      \end{shaded}

      \definecolor{shadecolor}{rgb}{0.74, 0.83, 0.9} % pale blue
      \begin{shaded}
        \textbf{Ideia:} remover uma aresta de nível $i$ da floresta $F_i$ , quebra uma componente desta em duas, $T_u$ e $T_v$.  Para isso, precisamos buscar por uma aresta substituta que tenha o menor peso e que ligue $T_u$ a $T_v$. Assim, em vez de usar um mapa hash para armazenar os vizinhos de cada vértice, usaremos um min-heap, onde a chave dessa estrutura de dados para um vizinho $v$ será o peso da aresta $uv$.
      \end{shaded}


    \end{block}

    \begin{block}{Informações e contato}
      Para mais informações, acesse a página do trabalho: \textcolor{jblue}{{\url{https://linux.ime.usp.br/~cjinshian/}}}

      \medskip
      Endereço para contato: \\ \textcolor{jblue}{{\url{cjinshian77@usp.br}}}
    \end{block}

    \begin{block}{Referências}
      \scriptsize{
      \begin{thebibliography}{30}

          \bibitem{jacob_holm}
          Holm, J., de Lichtenberg, K., Thorup, M.,
          ``\textbf{Poly-Logarithmic Deterministic Fully-Dynamic Algorithms for Connectivity, Minimum Spanning Tree, 2-Edge, and Biconnectivity},''
          \textit{Journal of the ACM}, 48(4): 723--760, 2001.

      \end{thebibliography}
      }


    \end{block}


  \end{columns}
\end{frame}
\end{document}