%!TeX root=../tese.tex
%("dica" para o editor de texto: este arquivo é parte de um documento maior)
% para saber mais: https://tex.stackexchange.com/q/78101

\chapter{Conexidade em florestas dinâmicas}

\section{Definição}

Rodrigues \cite{arthur}, em sua dissertação do mestrado, estudou, entre outros assuntos, o problema da conexidade em florestas dinâmicas e implementou o seu algoritmo, que foi proposto na Seção 2 do artigo de Holm, de Lichtenberg e Thorup \cite{jacob_holm}. No Capítulo 2 da sua dissertação ele descreve as rotinas principais do algoritmo, baseado em \textit{Euler tour trees}, e realiza uma análise minuciosa da complexidade de tempo das funções dos pseudocódigos. Dada essas circunstâncias, optamos por não apresentar uma descrição detalhada desse algoritmo, explicando brevemente sobre o que as rotinas principais fazem, bem como algumas diferenças da nossa implementação em código comparadas com a de Rodrigues.  

O problema da conexidade em florestas dinâmicas pode ser considerada uma simplificação do problema de conexidade em grafos dinâmicos, quando o grafo em questão é uma floresta. A biblioteca que usaremos contém os seguintes métodos:

\begin{itemize}
    \item \texttt{\textbf{florestaDinâmica(F, n)}}: constrói e devolve uma floresta dinâmica $F$ com $n$ vértices isolados;
    \item \texttt{\textbf{conectado(F, u, v)}}: devolve verdadeiro se \textit{u} e \textit{v} estão na mesma componente da floresta $F$ e falso caso contrário;
    \item \texttt{\textbf{ligue(F, u, v)}}: insere uma aresta \textit{uv} na floresta $F$;
    \item \texttt{\textbf{corte(u, v)}}: remove a aresta \textit{uv} da floresta $F$;
\end{itemize}

A estrutura de dados principal usada neste algoritmo de Holm, de Lichtenberg e Thorup para dar suporte eficientes às rotinas acima é uma árvore binária de busca balanceada (ABBB). Dessa forma, a floresta dinâmica é constituída de várias ABBBs. Rodrigues utiliza \textit{treaps} em sua implementação, que é de natureza aleatória. Em nosso caso, utilizamos árvores \textit{splay}, que foram desenvolvidas por \textit{Sleator e Tarjan} \cite{sleator}. Árvores \textit{splay} são árvores binárias de busca (ABBs) que possuem uma rotina extra (além das usuais de busca, inserção e remoção) chamada \textit{splay}, que é acionada ao final de cada operação feita na árvore, de modo que é sempre aplicada ao nó mais profundo visitado. Isso faz com que o custo amortizado da operação \textit{splay} seja $O(\lg n)$, onde $n$ é o número de nós da árvore, e que também uma sequência de $m$ acessos em uma árvore \textit{splay} tenha custo total $O(m \lg n)$.  Como também já existe bastante literatura sobre árvores \textit{splay}, e seu funcionamento interno não afeta a descrição dos algoritmos que descreveremos, não entraremos em detalhes de sua implementação.
