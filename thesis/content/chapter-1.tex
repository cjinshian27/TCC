%!TeX root=../tese.tex
%("dica" para o editor de texto: este arquivo é parte de um documento maior)
% para saber mais: https://tex.stackexchange.com/q/78101

\chapter{Conexidade em grafos dinâmicos}

\section{Introdução}

São considerados problemas de conexidade em grafos completamente dinâmicos aqueles em que o grafo sofre, com o tempo, alterações como inserções e remoções 
de arestas. Caso o algoritmo permita apenas inserção ou apenas remoção, tais 
problemas são chamados de parcialmente dinâmicos. Note que as operações de 
atualização e consulta são apresentadas de forma online, sem conhecimento das  operações futuras. 

Aqui será tratado problemas em que o grafo dinâmico possui um conjunto fixo de vértices \textit{V}, $|\textit{V}\ | = n$. Assim, pode-se definir $m$ como o número de arestas existentes, de modo que $m = 0$ antes de se realizar as consultas. Na maior parte das vezes, a complexidade de tempo das operações será amortizada, o que implica que eles são calculados como a média sobre todas as operações realizadas. 

Na implementação de Holm, Lichtenberg e Thorup






