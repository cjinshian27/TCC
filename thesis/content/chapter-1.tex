%!TeX root=../tese.tex
%("dica" para o editor de texto: este arquivo é parte de um documento maior)
% para saber mais: https://tex.stackexchange.com/q/78101

\chapter{Conexidade em florestas dinâmicas}

\section{Definição do problema}

O problema da conexidade em florestas dinâmicas pode ser considerada uma simplificação do problema de conexidade em grafos dinâmicos, quando o grafo em questão é uma floresta. Visto que já existe literatura descrevendo o primeiro problema de forma mais detalhada, como em Rodrigues \cite{arthur} no capítulo 2, aqui será feita apenas uma breve descrição de como o nosso algoritmo foi implementado. É importante ressaltar que a nossa implementação foi baseada na solução da seção 2 do artigo de \textit{Holm, de Lichtenberg e Thorup} \cite{jacob_holm}, porém escolhendo uma estrutura de dados apropriada para representar as árvores da floresta.

A nossa biblioteca contém os seguintes métodos:

\begin{itemize}
    \item \texttt{\textbf{florestaDinâmica(n)}}: constrói uma floresta dinâmica de $n$ vértices isolados;
    \item \texttt{\textbf{conectado(u, v)}}: devolve verdadeiro se \textit{u} e \textit{v} estão na mesma componente e falso caso contrário;
    \item \texttt{\textbf{ligue(u, v)}}: cria uma aresta \textit{uv} na floresta;
    \item \texttt{\textbf{corte(u, v)}}: remove a aresta \textit{uv} da floresta;
    \item \texttt{\textbf{imprimaArestas()}}: imprime todas as arestas da floresta na forma \textit{uv};
    \item \texttt{\textbf{imprimaÁrvores()}}: imprime todas as árvores da floresta, onde cada uma é identificada univocamente por um inteiro positivo.
\end{itemize}

Note que os métodos \texttt{imprimaArestas()} e \texttt{imprimaÁrvores()} não são considerados operações de modificação ou consulta, mas sim métodos auxiliares destinados à visualização do resultado da floresta após as operações realizadas, como também para a depuração do código.  

Utilizamos árvores \textit{splay}, que foram desenvolvidas por \textit{Sleator e Tarjan} \cite{sleator}. Árvores \textit{splay} são árvores binárias de busca (ABBs) que possuem uma rotina extra (além das usuais de busca, inserção e remoção) chamada \textit{splay}, que é acionada ao final de cada operação feita na árvore, de modo que é sempre aplicada ao nó mais profundo visitado. Assim, a árvore fica mais balanceada à medida que a acionamos mais devido às rotações que acontecem na árvore. Isso faz com que o custo amortizado da operação \textit{splay} seja $O(\lg n)$, e que também uma sequência de $m$ acessos em uma árvore \textit{splay} tenha custo amortizado $O(m \lg n)$.  Como também já existe bastante literatura sobre árvores \textit{splay}, então assumiremos a corretude desta estrutura de dados.  

As árvores da floresta guardam as trilhas eulerianas dos vértices, em inglês \textit{Euler Tour Trees}, que foram propostas por \textit{Tarjan e Vishkin} \cite{tarjan} em 1985, de modo que cada trilha representa uma árvore. O nome da técnica de construção da árvore a partir da trilha se chama \textit{Euler tour technique}, que é descrita no capítulo 2.1 da dissertação de \textit{Rodrigues} \cite{arthur}.   
