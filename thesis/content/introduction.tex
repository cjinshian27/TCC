%!TeX root=../tese.tex
%("dica" para o editor de texto: este arquivo é parte de um documento maior)
% para saber mais: https://tex.stackexchange.com/q/78101

%% ------------------------------------------------------------------------- %%

% "\chapter" cria um capítulo com número e o coloca no sumário; "\chapter*"
% cria um capítulo sem número e não o coloca no sumário. A introdução não
% deve ser numerada, mas deve aparecer no sumário. Por conta disso, este
% modelo define o comando "\chapter**".
\chapter{Introdução}
\label{cap:introducao}

\enlargethispage{.5\baselineskip}

Grafos são estruturas de dados que nos permitem modelar vários problemas existentes da vida real, sejam eles estáticos ou dinâmicos. Em problemas estáticos, isto é, o grafo não sofre alterações com o passar do tempo, podemos citar o planejamento de rotas de entrega, análise de moléculas químicas e de dependências em \textit{software} utilizando ordenação topológica. Entretanto, ainda existem muitas situações em que ocorre dinamicidade, como nas interações de usuários em redes sociais, monitoramento de epidemias (contatos e isolamentos) e sistemas de navegação \textit{GPS}, onde há necessidade de recalcular rotas dependendo das condições como congestionamentos e acidentes. Para tais problemas, usamos grafos dinâmicos para modelá-los.

Dessa forma, são considerados problemas em grafos completamente dinâmicos aqueles em que o grafo sofre, com o tempo, alterações como inserções e remoções 
de arestas. Caso o algoritmo permita apenas inserção ou apenas remoção, tais 
problemas são chamados de parcialmente dinâmicos, conforme \textit{Holm, de Lichtenberg e Thorup} \cite{jacob_holm}. Note que as operações de 
atualização e consulta são apresentadas de forma online, sem conhecimento das  operações futuras.

Aqui será tratado problemas em que o grafo dinâmico possui um conjunto fixo de vértices \textit{V}, $n = |\textit{V}\ |$. Além disso, pode-se definir $m$ como o número de arestas existentes, de modo que $m = 0$ antes de se realizar as consultas. Na maior parte das vezes, a complexidade de tempo das operações será amortizada, o que implica que elas são calculados como a média sobre todas as operações realizadas. 


Um grafo dinâmico de ordem \textbf{n} é uma sequência de grafos ($G_0$, $G_1$, ..., $G_T$), onde $G_0$ é um grafo com \textit{n} vértices e cada $G_t$ para $1 \leq t \leq T$ é obtido a partir de $G_{t-1}$ pela adição ou remoção de uma aresta. Assim, podemos escrever $E(G_{t}) := E(G_{t - 1}) \setminus {uv}$, para alguma aresta $uv \in E(G_{t-1})$. Chamamos de \textbf{alterações}, \textbf{modificações} ou \textbf{atualizações} quando ocorre alguma operação de adição e remoção de arestas no grafo dinâmico.

Um problema em grafos dinâmicos envolve verificar se o grafo atual \textit{G} satisfaz alguma propriedade, e cada operação que faz essa verificação é denominada \textbf{consulta}. A solução do problema depende da criação de um algoritmo que utiliza uma estrutura de dados capaz de realizar estas consultas de forma eficiente. 

Existem diversas famílias de grafos, de modo que cada uma é adequada para resolver um tipo de problema particular. Iremos tratar inicialmente do \textbf{problema de conexidade em grafos dinâmicos}, que consiste em manter um grafo dinâmico submetido a uma sequência de inserções e remoções de arestas. Entre essas modificações, é possível realizar consultas para verificar se dois vértices \textit{u} e \textit{v} estão conectados por algum caminho. Para este problema, iremos usar as \textbf{florestas} (coleção de árvores) como base para a construção do algoritmo.

Inicialmente, utilizaremos o \textbf{problema de conexidade em florestas dinâmicas} como base para construir o algoritmo do \textbf{problema de conexidade em grafos dinâmicos}, e, a partir este, construir o \textbf{algoritmo decremental para árvores geradoras de custo mínimo}. Tais problemas possuem soluções propostas por \textit{Holm, de Lichtenberg e Thorup} \cite{jacob_holm}, nas quais a nossa implementação será baseada, e usaremos \textit{C++} como linguagem do código, onde disponibilizamos no repositório do \textit{GitHub} \cite{chung2025}.