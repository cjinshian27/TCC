\chapter{Testes experimentais}
\label{chapter:experiments}

Nesta seção, apresentaremos alguns testes de comparação de performance entre os algoritmos de Kruskal e da MSF decremental. Os experimentos foram realizados em um computador pessoal com as seguintes configurações: 

\begin{itemize}
    \item Sistema operacional Ubuntu 24.04.3 LTS (kernel 6.8.0-49-generic);
    \item Processador Intel Core i5-8265U (4 núcleos, 8 threads), arquitetura x86\_64;
    \item Memória RAM de 7,6 GB; 
    \item GPU Intel UHD Graphics 620.
\end{itemize}

Para calcular a duração de tempo de uma operação em milissegundos, utilizamos a biblioteca \texttt{chrono} da linguagem \textit{C++}. Além disso, o experimento não considera o tempo para inicializar um grafo com $n$ vértices e $m$ arestas, apenas a remoção das arestas e a consulta do peso da MST após a remoção delas. 

Foi construído um gerador de grafos aleatórios, seguindo um modelo $G(n, p)$, em que $n$ é o número de vértices do grafo $G$ e $p$ a probabilidade da existência de cada aresta. Considera-se como limiar crítico o valor de $p = \frac{\lg n}{n} $. Assim, temos que o número esperado de arestas em um grafo $G(n, p)$ é dado por $\frac{n(n-1)}{2}p$. Ademais, a cada aresta é atribuído um peso inteiro no intervalo de $1$ a $n$.

Cada arquivo gerado de teste possui a primeira linha com os valores $n$ e $m$, seguida de $m$ linhas da forma $u$ $v$ $w$, onde $w$ é o peso da aresta $uv$. Em seguida, temos outras $m$ linhas da forma $u$ $v$, que representam uma permutação aleatória de arestas a serem removidas. No experimento, entretanto, removemos somente as primeiras 25.600 arestas da permutação, o que foi suficiente para demonstrar a eficiência do algoritmo da MSF decremental.

Cada grafo gerado possui $n = 20.000$ vértices, o que dá um número esperado de arestas de $142.000$. Nos nossos experimentos, geramos seis grafos $G_i$, $0 \leq i \leq 5$, com cinco permutações aleatórias distintas de arestas para cada.   
Inicialmente, avaliamos o grafo $G_0$ considerando a remoção de $25\,600$ arestas em cinco permutações distintas. Para cada permutação, registramos o peso da MST após a remoção de cada lote de arestas e o tempo de execução, em milissegundos, dos algoritmos de Kruskal e MSF Decremental. Depois, calculamos a média dos cinco testes realizados. Por fim, realizamos os mesmos experimentos de $G_0$  em outros cinco grafos, usando somente a primeira permutação de arestas gerada para cada grafo. Os resultados foram apresentados em tabelas e gráficos gerados por programas escritos em \textit{Python~3}.

Para cada grafo, realizamos a remoção de arestas em lotes sucessivos. Após cada lote de remoções, imprimimos a MST correspondente ao estado atual do grafo e, em seguida, dobramos o tamanho do lote. Dessa forma, removemos $100, 200, 400, \ldots, 25.600$ arestas, registrando tanto o tempo acumulado das remoções quanto o peso da MST após cada etapa.

\section{Algoritmo de Kruskal}

O algoritmo de Kruskal foi implementado de uma forma que permita remoção de arestas. Utilizamos Union-Find como a parte central do algoritmo. Na inicialização do construtor, ordenamos um vetor de $m$ arestas em ordem crescente de peso, usando o método \texttt{sort} da linguagem \textit{C++}, que consome tempo esperado $O(m \lg m)$. Em seguida, acionamos o método \texttt{union} para adicionar as arestas, uma a uma, em ordem crescente de peso.

Quando removemos uma aresta $uv$, precisamos reordenar as arestas restantes. Para isso, usamos o algoritmo de ordenação \textit{insertion sort}, visto que ele consome tempo $O(m)$ para ordenar um vetor de arestas quase ordenado. No nosso caso, como a $uv$ é trocada pela última aresta do vetor e removida, obtemos um vetor quase ordenado. Isso garante a complexidade de tempo $O(m)$ quando ordenamos as arestas remanescentes neste vetor.

\section{Grafo $G_0$}

O grafo $G_0$ foi gerado com 142.069 arestas, que é um número próximo do esperado. A seguir, mostraremos os resultados das remoções em cada permutação de arestas. É importante ressaltar que, como o algoritmo de Kruskal consome tempo $O(m)$ por remoção de aresta, optamos por fazer uma compressão de valores dos eixos $x$ e $y$ dos gráficos para evitar a sobreposição dos números, mas ainda manter um aspecto linear de crescimento.

\subsection{Permutação $0$ de arestas}

\begin{figure}[H]
    \centering
    \includegraphics[width=11cm]{./figuras/performance_table_graph_0_permutation_0.png}
    \caption{Desempenho dos algoritmos de Kruskal e MSF Decremental na permutação $0$ de $G_0$, exibindo o valor do peso da MST após a remoção de cada lote de arestas.}
    \label{fig:table_g0_permutation0}
\end{figure}

\begin{figure}[H]
    \centering
    \includegraphics[width=11cm]{./figuras/graph_0_permutation_0.png}
\caption{Desempenho dos algoritmos de Kruskal e MSF Decremental na permutação $0$ de $G_0$.}
    \label{fig:graph_g0_permutation0}
\end{figure}

\subsection{Permutação $1$ de arestas}

\begin{figure}[H]
    \centering
    \includegraphics[width=11cm]{./figuras/performance_table_graph_0_permutation_1.png}
    \caption{Desempenho dos algoritmos de Kruskal e MSF Decremental na permutação $1$ de $G_0$, exibindo o valor do peso da MST após a remoção de cada lote de arestas.}
    \label{fig:table_g0_permutation1}
\end{figure}

\begin{figure}[H]
    \centering
    \includegraphics[width=11cm]{./figuras/graph_0_permutation_1.png}
\caption{Desempenho dos algoritmos de Kruskal e MSF Decremental na permutação $1$ de $G_0$.}
    \label{fig:graph_g0_permutation1}
\end{figure}

\subsection{Permutação $2$ de arestas}

\begin{figure}[H]
    \centering
    \includegraphics[width=11cm]{./figuras/performance_table_graph_0_permutation_2.png}
    \caption{Desempenho dos algoritmos de Kruskal e MSF Decremental na permutação $2$ de $G_0$, exibindo o valor do peso da MST após a remoção de cada lote de arestas.}
    \label{fig:table_g0_permutation2}
\end{figure}

\begin{figure}[H]
    \centering
    \includegraphics[width=11cm]{./figuras/graph_0_permutation_2.png}
\caption{Desempenho dos algoritmos de Kruskal e MSF Decremental na permutação $2$ de $G_0$.}
    \label{fig:graph_g0_permutation2}
\end{figure}

\subsection{Permutação $3$ de arestas}

\begin{figure}[H]
    \centering
    \includegraphics[width=11cm]{./figuras/performance_table_graph_0_permutation_3.png}
    \caption{Desempenho dos algoritmos de Kruskal e MSF Decremental na permutação $3$ de $G_0$, exibindo o valor do peso da MST após a remoção de cada lote de arestas.}
    \label{fig:table_g0_permutation3}
\end{figure}

\begin{figure}[H]
    \centering
    \includegraphics[width=11cm]{./figuras/graph_0_permutation_3.png}
\caption{Desempenho dos algoritmos de Kruskal e MSF Decremental na permutação $3$ de $G_0$.}
    \label{fig:graph_g0_permutation3}
\end{figure}

\subsection{Permutação $4$ de arestas}

\begin{figure}[H]
    \centering
    \includegraphics[width=11cm]{./figuras/performance_table_graph_0_permutation_4.png}
    \caption{Desempenho dos algoritmos de Kruskal e MSF Decremental na permutação $4$ de $G_0$, exibindo o valor do peso da MST após a remoção de cada lote de arestas.}
    \label{fig:table_g0_permutation4}
\end{figure}

\begin{figure}[H]
    \centering
    \includegraphics[width=11cm]{./figuras/graph_0_permutation_4.png}
\caption{Desempenho dos algoritmos de Kruskal e MSF Decremental na permutação $4$ de $G_0$.}

    \label{fig:graph_g0_permutation4}
\end{figure}

\subsection{Tempo médio de execução das cinco permutações de arestas}

\begin{figure}[H]
    \centering
    \includegraphics[width=11cm]{./figuras/performance_table_graph_0_average.png}
\caption{Tempo médio de execução obtido a partir de cinco permutações de arestas de $G_0$.}
    \label{fig:table_g0_average}
\end{figure}

\begin{figure}[H]
    \centering
    \includegraphics[width=11cm]{./figuras/graph_0_average.png}
\caption{Comparação do tempo médio de execução dos algoritmos de Kruskal e MSF Decremental ao longo de cinco permutações de arestas de $G_0$.}
    \label{fig:graph_g0_average}
\end{figure}

\section{Grafos $G_1$, $G_2$, $G_3$, $G_4$ e $G_5$}

Neste experimento, testamos apenas a primeira permutação de arestas de cada grafo. O grafo $G_1$ foi gerado com $142.589$ arestas; o grafo $G_2$ com $142.954$ arestas; o grafo $G_3$ com $142.734$ arestas; o grafo $G_4$ com $142.402$ arestas; o grafo $G_5$ com $143.216$ arestas.

\subsection{Grafo $G_1$}

\begin{figure}[H]
    \centering
    \includegraphics[width=11cm]{./figuras/performance_table_graph_1.png}
    \caption{Desempenho dos algoritmos de Kruskal e MSF Decremental na permutação $0$ de $G_1$, exibindo o valor do peso da MST após a remoção de cada lote de arestas.}
    \label{fig:table_g1_permutation0}
\end{figure}

\begin{figure}[H]
    \centering
    \includegraphics[width=11cm]{./figuras/graph_1.png}
\caption{Desempenho dos algoritmos de Kruskal e MSF Decremental na permutação $0$ de $G_1$.}
    \label{fig:graph_g1_permutation0}
\end{figure}

\subsection{Grafo $G_2$}

\begin{figure}[H]
    \centering
    \includegraphics[width=11cm]{./figuras/performance_table_graph_2.png}
    \caption{Desempenho dos algoritmos de Kruskal e MSF Decremental na permutação $0$ de $G_2$, exibindo o valor do peso da MST após a remoção de cada lote de arestas.}
    \label{fig:table_g2_permutation0}
\end{figure}

\begin{figure}[H]
    \centering
    \includegraphics[width=11cm]{./figuras/graph_2.png}
\caption{Desempenho dos algoritmos de Kruskal e MSF Decremental na permutação $0$ de $G_2$.}
    \label{fig:graph_g2_permutation0}
\end{figure}

\subsection{Grafo $G_3$}

\begin{figure}[H]
    \centering
    \includegraphics[width=11cm]{./figuras/performance_table_graph_3.png}
    \caption{Desempenho dos algoritmos de Kruskal e MSF Decremental na permutação $0$ de $G_3$, exibindo o valor do peso da MST após a remoção de cada lote de arestas.}
    \label{fig:table_g3_permutation0}
\end{figure}

\begin{figure}[H]
    \centering
    \includegraphics[width=11cm]{./figuras/graph_3.png}
\caption{Desempenho dos algoritmos de Kruskal e MSF Decremental na permutação $0$ de $G_3$.}
    \label{fig:graph_g3_permutation0}
\end{figure}


\subsection{Grafo $G_4$}

\begin{figure}[H]
    \centering
    \includegraphics[width=11cm]{./figuras/performance_table_graph_4.png}
    \caption{Desempenho dos algoritmos de Kruskal e MSF Decremental na permutação $0$ de $G_4$, exibindo o valor do peso da MST após a remoção de cada lote de arestas.}
    \label{fig:table_g4_permutation0}
\end{figure}

\begin{figure}[H]
    \centering
    \includegraphics[width=11cm]{./figuras/graph_4.png}
\caption{Desempenho dos algoritmos de Kruskal e MSF Decremental na permutação $0$ de $G_4$.}
    \label{fig:graph_g4_permutation0}
\end{figure}

\subsection{Grafo $G_5$}

\begin{figure}[H]
    \centering
    \includegraphics[width=11cm]{./figuras/performance_table_graph_5.png}
    \caption{Desempenho dos algoritmos de Kruskal e MSF Decremental na permutação $0$ de $G_5$, exibindo o valor do peso da MST após a remoção de cada lote de arestas.}
    \label{fig:table_g5_permutation0}
\end{figure}

\begin{figure}[H]
    \centering
    \includegraphics[width=11cm]{./figuras/graph_5.png}
\caption{Desempenho dos algoritmos de Kruskal e MSF Decremental na permutação $0$ de $G_5$.}
    \label{fig:graph_g5_permutation0}
\end{figure}

\subsection{Tempo médio de execução dos cinco grafos}

\begin{figure}[H]
    \centering
    \includegraphics[width=11cm]{./figuras/performance_table_5_graphs_average.png}
\caption{Tempo médio de execução obtido a partir de cinco grafos.}
    \label{fig:performance_table_5_graphs_average}
\end{figure}

\begin{figure}[H]
    \centering
    \includegraphics[width=11cm]{./figuras/5_graphs_average.png}
\caption{Comparação do tempo médio de execução dos algoritmos de Kruskal e MSF Decremental considerando cinco grafos distintos.}
    \label{fig:5_graphs_average}
\end{figure}

\section{Conclusão}

A partir dos experimentos realizados, verifica-se que os grafos gerados para os testes apresentam um número de arestas próximo ao valor esperado para um grafo do modelo $G(n, p)$. Além disso, os resultados indicam que, nos gráficos de tempo de execução, o algoritmo de Kruskal apresenta um crescimento aproximadamente linear em função do número de arestas removidas, enquanto o algoritmo de MSF decremental exibe um padrão de crescimento logarítmico.

As tabelas de resultados, por outro lado, evidenciam a diferença significativa no tempo de execução entre os dois algoritmos durante o processo de remoção de arestas, diferença esta que já se mostra perceptível mesmo nas primeiras remoções realizadas.

Por fim, observa-se que os experimentos conduzidos sobre diferentes grafos com o mesmo número de vértices produzem gráficos e tabelas com comportamentos semelhantes no que tange ao tempo de execução, indicando consistência nos resultados obtidos.
