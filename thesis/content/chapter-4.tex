\chapter{Conclusões}
\label{chapter:conclusions}

Neste capítulo, consideraremos o problema da \textbf{MSF dinâmica}, onde queremos manter uma MSF em um grafo dinâmico. Estudaremos brevemente um algoritmo para MSF dinâmica. Como este algoritmo não foi implementado em nosso estudo, apresentaremos apenas a ideia por trás dele, de como podemos manter o peso mínimo de uma MSF de um grafo $G$ dando suporte eficiente à adição e remoção de arestas. Inicialmente, será descrito o funcionamento das \textbf{top trees}, estruturas de dados que serão usadas na manutenção de uma MSF dinâmica. Estas estruturas estão descritas na Seção~2.2 do artigo de Holm, de Lichtenberg e Thorup~\cite{jacob_holm}. 


Por fim, seria interessante apresentar os detalhes de como implementar a biblioteca deste algoritmo para MSF dinâmica. Para isso, entretanto, precisaríamos modificar a implementação dos algoritmo da conexidade em florestas dinâmicas, que passaria a usar top trees no lugar de Euler tour trees. Como a nossa implementação de Euler tour trees possui um consumo de tempo diferente das Top trees apresentadas em Holm, de Lichtenberg e Thorup~\cite{jacob_holm}, isso levaria à mudança na complexidade de tempo do algoritmo de conexidade em grafos dinâmicos, visto que este manteria estas florestas dinâmicas ajustadas. 

Além disso, ajustes no algoritmo de conexidade em grafos dinâmicos alteraria, por sua vez, a complexidade de tempo do algoritmo para MSF decremental.  

 conexidade em grafos dinâmicos e   