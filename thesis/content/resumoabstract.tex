%!TeX root=../tese.tex
%("dica" para o editor de texto: este arquivo é parte de um documento maior)
% para saber mais: https://tex.stackexchange.com/q/78101

% As palavras-chave são obrigatórias, em português e em inglês, e devem ser
% definidas antes do resumo/abstract. Acrescente quantas forem necessárias.
\palavraschave{grafo dinâmico, floresta geradora maximal de custo mínimo, splay trees}

\keywords{dynamic graph, minimum spanning forest, splay trees}

% O resumo é obrigatório, em português e inglês. Estes comandos também
% geram automaticamente a referência para o próprio documento, conforme
% as normas sugeridas da USP.
\resumo{
Grafos dinâmicos permitem modelar problemas em
que o conjunto de arestas do grafo sofre alterações
ao longo do tempo. Um dos problemas fundamentais nesse contexto é a manutenção de uma árvore geradora de custo mínimo do grafo dinâmico no decorrer de várias alterações no grafo. Neste trabalho, estudamos e implementamos vários algoritmos propostos por Holm, de Lichtenberg e Thorup~\cite{jacob_holm} para variantes desse problema. O foco foi no algoritmo para manter uma
floresta geradora maximal de custo mínimo (MSF) decremental, que dá suporte eficiente à remoção de arestas. Além disso, estudamos e apresentamos a ideia  do problema de manter uma floresta geradora maximal de custo mínimo (MSF) dinâmica, que dá suporte eficiente à adição e remoção de arestas.}

\abstract{
Dynamic graphs allow us to model problems in which the set of edges changes over time. One of the fundamental problems in this context is maintaining a minimum spanning tree of the dynamic graph as it undergoes multiple updates. In this work, we study and implement several algorithms proposed by Holm, de Lichtenberg, and Thorup~\cite{jacob_holm} for variants of this problem. Our main focus is the algorithm for maintaining a decremental minimum spanning forest, which efficiently supports edge deletions. In addition, we study and outline the approach for maintaining a fully dynamic minimum spanning forest, which efficiently supports both edge insertions and deletions.
}
