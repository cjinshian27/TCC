%!TeX root=../tese.tex
%("dica" para o editor de texto: este arquivo é parte de um documento maior)
% para saber mais: https://tex.stackexchange.com/q/78101

% As palavras-chave são obrigatórias, em português e em inglês, e devem ser
% definidas antes do resumo/abstract. Acrescente quantas forem necessárias.
\palavraschave{Grafo dinâmico, Floresta geradora maximal de custo mínimo, Splay trees}

\keywords{Dynamic graph, Minimum spanning forest, Splay trees}

% O resumo é obrigatório, em português e inglês. Estes comandos também
% geram automaticamente a referência para o próprio documento, conforme
% as normas sugeridas da USP.
\resumo{
Grafos dinâmicos permitem modelar problemas em
que o grafo sofre alterações ao longo do tempo. Um dos problemas fundamentais nesse contexto é a manutenção de uma árvore geradora de custo mínimo no decorrer de várias alterações no grafo. Neste trabalho, estudamos e implementamos vários algoritmos propostos por Holm, de Lichtenberg e Thorup para variantes desse problema. O foco foi no algoritmo para manter uma floresta geradora maximal de custo mínimo (MSF) decremental, em que se dá suporte eficiente à remoção de arestas do grafo. Além disso, estudamos as ideias usadas num algoritmo que mantém uma floresta geradora maximal de custo mínimo (MSF) em um grafo dinâmico, em que se dá suporte eficiente à adição e remoção de arestas.}

\abstract{
Dynamic graphs allow us to model problems in which the graph changes over time. One of the fundamental problems in this context is maintaining a minimum spanning tree of the dynamic graph as it undergoes multiple updates. In this work, we study and implement several algorithms proposed by Holm, de Lichtenberg, and Thorup for variants of this problem. Our main focus is the algorithm for maintaining a decremental minimum spanning forest, which efficiently supports edge deletions in the graph. In addition, we also study the approach for maintaining a fully dynamic minimum spanning forest, which efficiently supports both edge insertions and deletions in the graph.
}
