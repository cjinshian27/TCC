%!TeX root=../tese.tex
%("dica" para o editor de texto: este arquivo é parte de um documento maior)
% para saber mais: https://tex.stackexchange.com/q/78101

\chapter{Algoritmo para MSF decremental}

\enlargethispage{.8\baselineskip}

Na literatura, existe um problema famoso que é encontrar uma árvore geradora de custo mínimo (MST, de \textit{Minimum Spanning Tree}) para um dado grafo $G$ com $n$ vértices e $m$ arestas ponderadas. Como mencionado brevemente no Capítulo~\ref{cap:introducao}, o algoritmo de Kruskal é eficiente não só na criação da árvore geradora de custo mínimo, como também na consulta de teste de conectividade. Para este último, podemos aproveitar a estrutura do Union-Find para testar se dois vértices $u$ e $v$ estão conectados em, praticamente, tempo constante.

Apesar de o algoritmo de Prim ser eficiente para criação da MST de um grafo, a estrutura de dados usada dificulta a criação de uma rotina que teste a conectividade entre dois vértices do grafo, e por isso usaremos o algoritmo de Kruskal como referência a partir de agora.

Geralmente, usamos o algoritmo de Kruskal (e o de Prim também) em grafos que não sofrem adição ou remoção de arestas com o passar do tempo, visto que queremos saber apenas o custo da MST. Sendo assim, como em Kruskal não há suporte para alterações no grafo, toda vez que removemos uma aresta temos que reconstruir a MST correspondente ao grafo novo. Pelo fato de a ordenação das $m$ arestas de $G$ consumir $\Oh(m \lg m)$ e ser a parte mais custosa do algoritmo de Kruskal, então sempre que removermos alguma aresta, o algoritmo terá que reordenar as arestas restantes e construir um novo MST para $G$.

Sendo assim, para $k$ remoções de arestas, teremos que ordenar $k$ vezes, com o custo total sendo $\Oh(m \lg m) + \Oh((m - 1) \lg (m - 1)) + \ldots + \Oh((m - k) \lg (m - k))$, pois o número de arestas vai diminuindo com as remoções. Assim, se quisermos remover todas as $m$ arestas, podemos expressar o custo total das remoções no algoritmo de Kruskal pelo seguinte somatório:

\begin{equation}
    \sum_{x=1}^{m} x \lg x \approx \int_{1}^{m} x \lg x dx
    \label{first-equation}
\end{equation}
onde podemos escrever a integral como

\begin{equation}
    \int_{1}^{m} x \lg x dx = \dfrac{m^2}{2}\lg m - m^2 + \dfrac{1}{4}.
    \label{second-equation}
\end{equation}
Como o termo dominante em \ref{second-equation} é $\dfrac{m^2}{2}\lg m$, então temos que a remoção de todas as $m$ arestas possui um custo de tempo $\Oh(m^2 \lg m)$. 

Como remover arestas e ao mesmo tempo manter a MST de um grafo com as arestas restantes é ineficiente no algoritmo de Kruskal, introduziremos uma solução eficiente para este problema: o algoritmo decremental para florestas geradoras maximais de custo mínimo (MSF, de \textit{Minimum Spanning Forest}), que está descrito na Seção 4 do artigo de Holm, de Lichtenberg e Thorup~\cite{jacob_holm}.

Como o nome já sugere, o algoritmo decremental suporta apenas remoção de arestas quando se trata de alteração do grafo. A construção dele é baseada no algoritmo do grafo dinâmico apresentado na Seção~\ref{sec:dynamic-graph-creation}, com alguns ajustes extras a serem feitos. Por questões de simplicidade, abreviaremos o nome do algoritmo decremental para \textbf{MSF decremental}, que vem do inglês \textit{Minimum Spanning Forests}, e o descreveremos em mais detalhes nas seções seguintes.

\section{Biblioteca do MSF decremental}
\label{sec:decremental-msf-library}

Implementar o algoritmo decremental para florestas geradoras maximais de custo mínimo (MSF, de \textit{Minimum Spanning Forest}) resume-se à construção da seguinte biblioteca de forma eficiente:

\begin{itemize}
    \item \texttt{\textbf{MSFDecremental(n)}}: contrói e devolve um grafo $G$ com $n$ vértices e sem arestas;
    \item \texttt{\textbf{conectadosMSF(G, u, v)}}: devolve verdadeiro se os vértices $u$ e $v$ estão na mesma componente de $G$ e falso caso contrário;
    \item \texttt{\textbf{removaMSF(G, u, v)}}: remove a aresta $uv$ do grafo $G$.
\end{itemize} 

Note que, diferente da biblioteca do algoritmo do grafo dinâmico, apresentada na Seção~\ref{sec:dynamic-graph-routines}, no MSF decremental não temos um método equivalente a \texttt{adicioneGD} disponível para o usuário. Em nossa implementação \cite{chung2025}, para a entrada de um grafo $G$ de $n$ vértices e $m$ arestas ponderadas, inicializamos um grafo de $n$ vértices isolados chamando \texttt{MSFDecremental(n)}, e em seguida ordenaremos essas $m$ arestas em ordem crescente de peso, usando uma biblioteca pronta do $\emph{C++}$, que consome tempo esperado $\Oh(m \lg m)$. Em seguida, acionaremos \texttt{adicioneMSF}, que é uma rotina privada, e ela inserirá todas as $m$ arestas em ordem crescente de peso. 

Para o usuário, então, só estarão disponíveis as rotinas \texttt{conectadosMSF} e \texttt{removaMSF} como consultas ao grafo. A versão totalmente dinâmica, que inclui a rotina \texttt{adicioneMSF} para o usuário, será estudada posteriormente na Seção~\ref{sec:fully-MSF}.

O construtor \texttt{MSFDecremental} e a rotina \texttt{conectadosMSF} possuem, respectivamente, a mesma complexidade de tempo das rotinas \texttt{grafoDinâmico} e \texttt{conectadosGD} do algoritmo do grafo dinâmico. Por isso, passaremos brevemente sobre eles, e detalharemos mais a rotina \texttt{removaMSF}, que possui a rotina auxiliar \texttt{substituaAresta} implementada de maneira diferente do algoritmo do grafo dinâmico. 

Usaremos várias definições já apresentadas no algoritmo do grafo dinâmico, incluindo as mesmas invariantes apresentadas na Seção~\ref{sec:level-slicing}, os mesmos tipos de arestas da Seção~\ref{sec:dynamic-graph-edge-types} e nós das florestas apresentados na Seção~\ref{sec:graph-nodes}. A seguir, apresentaremos as rotinas do MSF decremental e alguns ajustes a serem feitos já que estamos nos baseando nas rotinas da biblioteca do grafo dinâmico. 

\subsection{Listas de Adjacências}
\label{sec:adjancency-lists-min-heap}

Na Seção~\ref{sec:dynamic-graph-edge-types}, apresentamos a biblioteca de \texttt{listasDeAdjacências}, onde usamos um mapa hash para inserir ou remover um vértice $v$ da lista de $u$, além de percorrer os vizinhos da lista de $u$. No algoritmo do MSF decremental, quando removemos uma aresta de nível $i$ da floresta $F_i$, uma componente desta será quebrada em duas, $T_u$ e $T_v$, da mesma forma que no algoritmo do grafo dinâmico. A diferença é que, no caso do MSF decremental, precisamos buscar por uma aresta substituta que tenha o menor peso e que ligue $T_u$ a $T_v$. Não podemos simplesmente percorrer todos os vizinhos $v$ de cada vértice $u$ em $T_u$, verificar se $uv$ reconecta as componentes separadas e se é de menor peso dentre todas as substitutas, já que isso seria ineficiente.

Assim, fica claro que precisamos percorrer as arestas reservas em ordem crescente de peso e testar se alguma é substituta. Por isso, em vez de usar um mapa hash para mapear os vizinhos das listas de adjacências de cada vértice, usaremos um min-heap. Na verdade, como estamos trabalhando com nós de vértice e de aresta, cada nó de vértice $u$ guardará um min-heap onde a chave dessa estrutura de dados será o peso do nó de aresta $uv$ e $v$ é o vizinho da lista de $u$. Nós de aresta também guardarão um min-heap, porém vazio. 

A biblioteca do min-heap está descrita a seguir.

\begin{itemize}
    \item \texttt{\textbf{MinHeap(n)}}: cria um min-heap vazio (em nossa implementação, criamos um vetor vazio);
    \item \texttt{\textbf{éVazio()}}: devolve verdadeiro se min-heap está vazio e falso caso contrário;
    \item \texttt{\textbf{tamanho()}}: retorna a quantidade de vizinhos de $v$ no min-heap;
    \item \texttt{\textbf{remova(u, v)}}: remove o vizinho $v$ do min-heap de $u$;
    \item \texttt{\textbf{consulteMínimo()}}: retorna um par $\{u, p\}$, onde $u$ é o vizinho de $v$ e $p$ é o menor peso dentre todas as arestas $uv$;
    \item \texttt{\textbf{extraiaMínimo(v)}}: remove o par $\{u, p\}$ do min-heap de $v$ e retorna este par;
    \item \texttt{\textbf{insira(u, v, p)}}: insere o vizinho $v$ no min-heap de $u$ e o peso $p$ da aresta reserva $uv$.
\end{itemize} 

Os métodos \texttt{remova}, \texttt{extraiaMínimo} e \texttt{insira} consomem tempo O($\lg n)$, visto que precisamos ajustar a posição dos vizinhos de $v$ no min-heap em ordem de peso após inserirmos ou removermos um vizinho. O resto dos métodos consomem tempo constante, e eles serão necessários para buscar a aresta substituta de peso mínimo, como descreveremos mais à frente.

Com base na biblioteca do min-heap, podemos definir a biblioteca das listas de adjacências do MSF decremental. 

\begin{itemize}
    \item \texttt{\textbf{listasDeAdjacênciasMinHeap(n)}}: constrói e devolve um grafo com $n$ vértices e sem arestas, representado por listas de adjacências;
    \item \texttt{\textbf{adicioneLAMinHeap(R, u, v, p)}}: adiciona o vértice $u$ na lista de adjacências de $v$ em $R$ e vice-versa, além do peso $p$ de $uv$;
    \item \texttt{\textbf{removaLAMinHeap(R, u, v)}}: remove o vértice $u$ da lista de adjacências de $v$ em $R$ e vice-versa.
\end{itemize} 

A chamada ao método \texttt{adicioneLAMinHeap(R, u, v, p)} aciona \texttt{insira(u, v, p)} e \texttt{insira(v, u, p)}, consumindo tempo $\Oh(\lg n)$. Similarmente, uma chamada ao método \texttt{removaLAMinHeap(R, u, v)} aciona \texttt{remova(u, v)} e \texttt{remova(v, u)}, consumindo tempo $\Oh(\lg n)$.

Como o min-heap é uma estrutura de dados bastante conhecida, não iremos descrever a sua implementação em detalhes. O objetivo é ressaltar as diferenças entre as listas de adjacências utilizadas no algoritmo do grafo dinâmico e MSF decremental, e como essa mudança afetará o comportamento do método \texttt{substituaAresta} do MSF decremental.

\section{Rotinas da biblioteca do MSF decremental}

\subsection{Criação do grafo}

A rotina \texttt{MSFDecremental}, como se pode ver no Programa~\ref{prog:newDecrementalMSF}, é bem parecida com a do grafo dinâmico, descrita na Seção~\ref{sec:dynamic-graph-creation}. 

\begin{programruledcaption}{\texttt{MSFDecremental($n$)} \label{prog:newDecrementalMSF}}
    \noindent\textbf{Entrada}: Recebe o número $n$ de vértices do grafo. \\
    \textbf{Saída}: Devolve um grafo $G$ com $n$ vértices e sem arestas.
    \vspace{-0.5\baselineskip}
    \begin{lstlisting}[
        language={[brazilian]pseudocode},
        style=pseudocode,
        style=wider,
        functions={},
        specialidentifiers={},
        escapeinside={(*@}{@*)},
    ]
    \textbf{para} $i$ := $1$ \textbf{até} $\left\lceil \lg n \right\rceil$ \textbf{faça}
        $G.F_i$ := \texttt{florestaDinâmica($n$)}
        $G.R_i$ := \texttt{listasDeAdjacênciasMinHeap($n$)}
    G.nível := \texttt{novoMapaHash($n$)}
    devolva G
    \end{lstlisting}
    \vspace{-0.5\baselineskip}
\end{programruledcaption}

Apesar das semelhanças, o construtor das listas de adjacências agora é implementado de forma diferente. Agora estamos usando o método \texttt{listasDeAdjacênciasMinHeap}. Como estamos apenas construindo 

Este método será descrito na Seção~\ref{sec:edge-addition-DecrementalMSF}. 

\section{Adição de arestas}
\label{sec:edge-addition-DecrementalMSF}
