%!TeX root=../tese.tex
%("dica" para o editor de texto: este arquivo é parte de um documento maior)
% para saber mais: https://tex.stackexchange.com/q/78101

\chapter{Algoritmo para MSF decremental}

\enlargethispage{.8\baselineskip}



\section{Biblioteca do MSF decremental}
\label{sec:decremental-msf-library}

Implementar o algoritmo decremental para florestas geradoras maximais de custo mínimo (MSF, de \textit{Minimum Spanning Forest}) resume-se à construção da seguinte biblioteca de forma eficiente:

\begin{itemize}
    \item \texttt{\textbf{MSFDecremental(n)}}: contrói e devolve um grafo $G$ com $n$ vértices e sem arestas;
    \item \texttt{\textbf{conectadosMSF(G, u, v)}}: devolve verdadeiro se os vértices $u$ e $v$ estão na mesma componente de $G$ e falso caso contrário;
    \item \texttt{\textbf{removaMSF(G, u, v)}}: remove a aresta $uv$ do grafo $G$.
\end{itemize} 

Note que, diferente do algoritmo do grafo dinâmico, apresentado na Seção~\ref{sec:dynamic-graph-routines}, não temos um \texttt{adicioneMSF} neste algoritmo, que vamos passar a chamar de \textbf{MSF decremental}, por questões de simplicidade. A versão totalmente dinâmica para florestas geradoras maximais de custo mínimo será estudada posteriormente na Seção~\ref{sec:fully-MSF}.

