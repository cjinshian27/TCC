%!TeX root=../tese.tex
%("dica" para o editor de texto: este arquivo é parte de um documento maior)
% para saber mais: https://tex.stackexchange.com/q/78101

\chapter{Conexidade em grafos dinâmicos}

\enlargethispage{.8\baselineskip}

\section{Definição}

Como citado no Capítulo 1, o problema da conexidade em grafos dinâmicos visa construir um algoritmo eficiente para a seguinte biblioteca, que contém:

\begin{itemize}
    \item \texttt{\textbf{grafoDinâmico(G, n)}}: devolve um grafo dinâmico $G$ com $n$ vértices isolados;
    \item \texttt{\textbf{conectado(G, u, v)}}: devolve verdadeiro se os vértices $u$ e $v$ estão na mesma componente de $G$ e falso caso contrário;
    \item \texttt{\textbf{adiciona(G, u, v)}}: adiciona a aresta $uv$ no grafo $G$;
    \item \texttt{\textbf{remova(G, u, v)}}: remove a aresta $uv$ do grafo $G$.
\end{itemize} 

O algoritmo de Holm, de Lichtenberg e Thorup para este problema de conexidade é composto por $\left\lceil \lg n \right\rceil$ florestas dinâmicas do grafo $G$, que utilizam o algoritmo mencionado na Seção 2. Cada aresta do grafo possui um nível entre $0$ e $\left\lceil \lg n \right\rceil$. O nível inicial de uma aresta recém-inserida é sempre $\left\lceil \lg n \right\rceil$, e ele nunca aumenta, apenas diminui. Assim, cada aresta de nível $i$ pertence à floresta dinâmica de mesmo nível. 

Ademais, a consulta \texttt{conectado(u, v)} aplicada ao grafo $G$ significa fazer a mesma consulta para alguma floresta maximal $F$ de $G$. Dessa maneira, sempre que estivermos realizando alguma operação de alteração ou consulta de conexidade em nosso grafo $G$, estamos realizando-a em uma floresta dinâmica $F$ que seja maximal em $G$. Da mesma forma, quando chamamos o construtor do grafo dinâmico, estamos criando $\lg n$ florestas de vértices isolados.