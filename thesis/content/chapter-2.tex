%!TeX root=../tese.tex
%("dica" para o editor de texto: este arquivo é parte de um documento maior)
% para saber mais: https://tex.stackexchange.com/q/78101

\chapter{Conexidade em grafos dinâmicos}

\enlargethispage{.8\baselineskip}

\section{Definição do problema}

Como citado na Seção 1.1, o problema de conexidade dos grafos dinâmicos será o passo intermediário para a definição do próximo problema de conectividade. A nossa biblioteca contém o seguinte:

\begin{itemize}
    \item \texttt{\textbf{grafoDinâmico(n)}}: constrói um grafo dinâmico de $n$ vértices;
    \item \texttt{\textbf{conectado(u, v)}}: devolve verdadeiro se os vértices $u$ e $v$ estão na mesma componente e falso caso contrário.
    \item \texttt{\textbf{adiciona(u, v)}}: adiciona a aresta $uv$ no grafo;
    \item \texttt{\textbf{remova(u, v)}}: remove a aresta $uv$ do grafo.
    \item \texttt{\textbf{imprima()}}: imprime todas os níveis das florestas e suas respectivas árvores.
\end{itemize}

Assim como o problema da conexidade em florestas dinâmicas, o método \texttt{imprima()} é usada apenas para visualização do resultado da floresta após as operações realizadas, como também para a depuração do código. 